\chapter{统计}

\section{统计图表}
\subsection{频率分布直方图}
\subsection{频率分布折线图}
\subsection{散点图}

\section{统计量}
在之前,我们已经学过了用均值和方差等统计量来估计总体的数字特征。现在,我们将复习已学的统计量并引入新的统计量。

\subsection{均值}
由于要表示大量数据的加法,我们引入求和符号$\Sigma$(读作sigma),${\displaystyle\sum_{i=0}^n a_i}$表示$a$从下标$i=0$到$i=n$的累加。

故此,均值公式可表示为
\begin{gather}
	\bar{x}=\frac{1}{n}\sum_{i=0}^n x_i \label{equ:mean-1}
\end{gather}

\subsection{方差和标准差}
\subsection{中位数和众数}
\subsection{百分位数}
\subsection{随机变量的分布特征}
