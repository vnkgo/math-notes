\chapter{三角比}
在初中时期,我们通过直角三角形定义并明白了锐角三角比。现在,我们将要学习任意角的三角比。

\section{概念}

\subsection{弧度制}
在之前很长一段时间,我们使用的都是角度制,它将一个周角分为了360等分,不过继续运用角度制会显得不够简洁。所以我们引入了\textbf{弧度}(radian),它认为一个周角是$2\pi$弧度。

从角度制转换为弧度制很简单,乘以$\dfrac{\pi}{180}$即可;以弧度表示角度,则乘以$\dfrac{180}{\pi}$,由此可知$1\mathrm{rad}=57.296^\circ$。

弧度的单位写作$\mathrm{rad}$,不过通常不写单位。

\subsection{三角比的定义}
我们用如图\ref{fig:unit-circle}的单位圆来定义任意角的三角比。虽然我们放弃了使用直角三角形定义三角比,但图中绿色部分仍然是一个三角形,只不过两条直角边(蓝边、红边)是要考虑正负的。

与锐角三角比类似,我们可以定义终边$r$与$x$轴正半轴夹角$\theta$的三角比

\begin{figure}[htb]
	\centering
    \begin{asy}
        write("asy: Generating 'fig:unit-circle'");
        size(6cm, 6cm);

        draw((-1.2, 0) -- (1.2, 0), Arrow, L=Label("$x$", position=EndPoint));
        draw((0, -1.2) -- (0, 1.2), Arrow, L=Label("$y$", position=EndPoint));
        draw(unitcircle);
        draw((0, 0) -- (1/2, sqrt(3)/2), linewidth(1.5), L=Label("$r$", position=MidPoint, align=N));
        fill((0, 0) -- (1/2, 0) -- (1/2, sqrt(3)/2) -- cycle, palegreen);
        draw((0, 0) -- (0.5, 0), blue+1);
        draw((1/2, 0) -- (1/2, sqrt(3)/2), red+1);
        draw(arc((0, 0), r=0.3, angle1=0, angle2=60), arrow=Arrow(TeXHead), L=Label("$\theta$", position=Relative(0.3), align=W));

        label("$P$", (1/2, sqrt(3)/2), align=NE);
    \end{asy}
	\caption{单位圆}
	\label{fig:unit-circle}
\end{figure}

\section{公式}
在以下4个小节中,我们将了解4类在三角学中很有用的公式。

\subsection{变名公式}
当角度不变,只改变三角比的名称时,应该使用这组公式。

第一部分由3个含平方的等式构成
\begin{gather}
	\sin^2\alpha+\cos^2\alpha=1 \label{equ:trig-1} \\
	\tan^2\alpha+1=\sec^2\alpha \label{equ:trig-2}\\
	1+\cot^2\alpha=\csc^2\alpha \label{equ:trig-3}
\end{gather}

第二部分由4个分式组成
\begin{gather}
	\tan\alpha=\frac{\sin\alpha}{\cos\alpha} \label{equ:trig-4} \\
	\cot\alpha=\frac{1}{\tan\alpha} \label{equ:trig-5} \\
	\sec\alpha=\frac{1}{\cos\alpha} \label{equ:trig-6} \\
	\csc\alpha=\frac{1}{\sin\alpha} \label{equ:trig-7}
\end{gather}

使用以上公式要考虑结果的正负,我们也可以使用直角三角形变名。

\subsection{变角公式}
变角公式有如下常用的5个小类。

\paragraph{和差角公式}
$\sin$和$\cos$的和差角公式是最基础的,以下所有公式都由它俩推得
\begin{gather}
	\sin(\alpha\pm\beta)=\sin\alpha\cos\beta\pm\cos\alpha\sin\beta \label{equ:trig-8} \\
	\cos(\alpha\pm\beta)=\cos\alpha\cos\beta\mp\sin\alpha\sin\beta \label{equ:trig-9}
\end{gather}

用式子\eqref{equ:trig-8}除以式子\eqref{equ:trig-9},每项同除$\cos\alpha\cos\beta$,可得到$\tan$的和差角公式
\begin{gather}
	\tan(\alpha\pm\beta)=\frac{\tan\alpha\pm\tan\beta}{1\mp\tan\alpha\tan\beta}\label{equ:trig-10}
\end{gather}

\paragraph{倍角公式}
通过将和差角公式中的$\alpha\pm\beta$替换成$\alpha+\alpha=2\alpha$,可得到倍角公式

\paragraph{降幂公式}
\paragraph{半角公式}
\paragraph{万能公式}

\subsection{诱导公式}
\subsection{辅助角公式}

\section{化简}
