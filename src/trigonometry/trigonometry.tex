\chapter{三角比}
在初中时期,我们通过直角三角形定义并明白了锐角三角比。现在,我们将要学习任意角的三角比。

\section{概念}

\subsection{弧度制}
在之前很长一段时间,我们使用的都是角度制,它将一个周角分为了360等分,不过继续运用角度制会显得不够简洁。所以我们引入了\textbf{弧度}(radian),它认为一个周角是$2\pi$弧度。

从角度制转换为弧度制很简单,乘以$\dfrac{\pi}{180}$即可;以弧度表示角度,则乘以$\dfrac{180}{\pi}$,由此可知$1\mathrm{rad}=57.296^\circ$。

弧度的单位写作$\mathrm{rad}$,不过通常不写单位。

\subsection{三角比的定义}
我们用如图\ref{fig:unit-circle}的单位圆来定义任意角的三角比。

\begin{figure}[htb]
	\centering
    \begin{asy}
        write("asy: Generating 'fig:unit-circle'");
        size(6cm, 6cm);

        draw((-1.2, 0) -- (1.2, 0), Arrow, L=Label("$x$", position=EndPoint));
        draw((0, -1.2) -- (0, 1.2), Arrow, L=Label("$y$", position=EndPoint));
        draw(unitcircle);
        draw((0, 0) -- (1/2, sqrt(3)/2), linewidth(1.5), L=Label("$r$", position=MidPoint, align=N));
        fill((0, 0) -- (1/2, 0) -- (1/2, sqrt(3)/2) -- cycle, palegreen);
        draw((0, 0) -- (0.5, 0), blue+1);
        draw((1/2, 0) -- (1/2, sqrt(3)/2), red+1);
        draw(arc((0, 0), r=0.3, angle1=0, angle2=60), arrow=Arrow(TeXHead), L=Label("$\theta$", position=Relative(0.3), align=W));

        label("$P$", (1/2, sqrt(3)/2), align=NE);
    \end{asy}
	\caption{单位圆}
	\label{fig:unit-circle}
\end{figure}

\section{公式}

\section{化简}
