\chapter{三角比}
在初中时期,我们通过直角三角形定义并明白了锐角三角比。现在,我们将要学习任意角的三角比。

\section{概念}

\subsection{弧度制}
在之前很长一段时间,我们使用的都是角度制,它将一个周角分为了360等分,不过继续运用角度制会显得不够简洁。所以我们引入了\textbf{弧度}(radian),它认为一个周角是$2\pi$弧度。

从角度制转换为弧度制很简单,乘以$\dfrac{\pi}{180}$即可;以弧度表示角度,则乘以$\dfrac{180}{\pi}$,由此可知$1\mathrm{rad}=57.296^\circ$。

弧度的单位写作$\mathrm{rad}$,不过通常不写单位。

\subsection{三角比的定义}
在过去我们使用直角三角形一个锐角的对边、邻边和斜边定义一个锐角的三角比。

\begin{figure}[htb]
	\centering
	\begin{asy}
		write("Creating 'fig:unit-circle'");
		size(6cm, 6cm);

		draw((-1.5, 0) -- (1.5, 0), Arrow, L=Label("$x$", position=EndPoint));
		draw((0, -1.5) -- (0, 1.5), Arrow, L=Label("$y$", position=EndPoint));
		draw(unitcircle);
	\end{asy}
	\caption{单位圆}
	\label{fig:unit-circle}
\end{figure}

\section{公式}

\section{化简}

