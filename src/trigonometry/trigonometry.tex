\chapter{三角比}
在初中时期,我们通过直角三角形定义并明白了锐角三角比。现在,我们将要学习任意角的三角比。

\section{概念}

\subsection{弧度制}
在之前很长一段时间,我们使用的都是角度制,它将一个周角分为了360等分,不过继续运用角度制会显得不够简洁。所以我们引入了\textbf{弧度}(radian),它认为一个周角是$2\pi$弧度。

从角度制转换为弧度制很简单,乘以$\dfrac{\pi}{180}$即可;以弧度表示角度,则乘以$\dfrac{180}{\pi}$,由此可知$1\mathrm{rad}=57.296^\circ$。

弧度的单位写作$\mathrm{rad}$,不过通常不写单位。

\subsection{三角比的定义}
我们用如图\ref{fig:unit-circle}的单位圆来定义任意角的三角比。虽然我们放弃了使用直角三角形定义三角比,但图中绿色部分仍然是一个三角形,只不过两条直角边(蓝边、红边)是要考虑正负的。

\begin{figure}[htb]
	\centering
    \begin{asy}
        write("asy: Generating 'fig:unit-circle'");
        size(6cm, 6cm);

        draw((-1.2, 0) -- (1.2, 0), Arrow, L=Label("$x$", position=EndPoint));
        draw((0, -1.2) -- (0, 1.2), Arrow, L=Label("$y$", position=EndPoint));
        draw(unitcircle);
        draw((0, 0) -- (1/2, sqrt(3)/2), linewidth(1.5), L=Label("$r$", position=MidPoint, align=N));
        fill((0, 0) -- (1/2, 0) -- (1/2, sqrt(3)/2) -- cycle, palegreen);
        draw((0, 0) -- (0.5, 0), blue+1);
        draw((1/2, 0) -- (1/2, sqrt(3)/2), red+1);
        draw(arc((0, 0), r=0.3, angle1=0, angle2=60), arrow=Arrow(TeXHead), L=Label("$\theta$", position=Relative(0.3), align=W));

        label("$P$", (1/2, sqrt(3)/2), align=NE);
    \end{asy}
	\caption{单位圆}
	\label{fig:unit-circle}
\end{figure}

与锐角三角比类似,我们可以定义终边$r$与$x$轴正半轴夹角$\theta$的三角比
\begin{align*}
    \sin\theta=\frac{y}{r}\quad & \cos\theta=\frac{x}{r} \\
    \tan\theta=\frac{y}{x}\quad & \cot\theta=\frac{x}{y} \\
    \sec\theta=\frac{r}{y}\quad & \csc\theta=\frac{r}{x}
\end{align*}

不难看出,当$r=1$时,红线的长度即$\theta$角的正弦值,称为\textbf{正弦线};同样,蓝线的长度是$\theta$角的余弦值,为\textbf{余弦线}。

所以,圆周上一点$P$可表示为\[P(r\cos\theta,r\sin\theta)\]

至于各个特殊角的三角比的值嘛,也是可以轻易算出来的,限于篇幅就不放在这里了,反正各位的教科书上也是有的。

\section{公式}
在以下4个小节中,我们将了解4类在三角学中很有用的公式(你我肯定有同感:“这玩意真多啊”,不过就20几个嘛,背就完事了)。

“三角比”和“三角函数”没有什么本质差别,所以这里直接懒得管了,将三角比和三角函数混用,请不要介意。

\subsection{变名公式}
当角度不变,只改变三角比的名称时,应该使用这组公式。

第一部分由3个含平方的等式构成
\begin{gather}
	\sin^2\alpha+\cos^2\alpha=1 \label{equ:trig-1} \\
	\tan^2\alpha+1=\sec^2\alpha \label{equ:trig-2}\\
	1+\cot^2\alpha=\csc^2\alpha \label{equ:trig-3}
\end{gather}

第二部分由4个分式组成
\begin{gather}
	\tan\alpha=\frac{\sin\alpha}{\cos\alpha} \label{equ:trig-4} \\
	\cot\alpha=\frac{1}{\tan\alpha} \label{equ:trig-5} \\
	\sec\alpha=\frac{1}{\cos\alpha} \label{equ:trig-6} \\
	\csc\alpha=\frac{1}{\sin\alpha} \label{equ:trig-7}
\end{gather}

使用以上公式要考虑结果的正负,我们也可以使用直角三角形变名。

\subsection{变角公式}
变角公式有如下常用的5类(“看到这么多公式不想背”的心情我能理解,不过这玩意不背三角学可是学不下去的)。

\paragraph{和差角公式}
$\sin$和$\cos$的和差角公式是最基础的,以下所有公式都由它俩推得
\begin{gather}
	\sin(\alpha\pm\beta)=\sin\alpha\cos\beta\pm\cos\alpha\sin\beta \label{equ:trig-8} \\
	\cos(\alpha\pm\beta)=\cos\alpha\cos\beta\mp\sin\alpha\sin\beta \label{equ:trig-9}
\end{gather}
用式子\eqref{equ:trig-8}除以式子\eqref{equ:trig-9},每项同除$\cos\alpha\cos\beta$,可得到$\tan$的和差角公式
\begin{gather}
	\tan(\alpha\pm\beta)=\frac{\tan\alpha\pm\tan\beta}{1\mp\tan\alpha\tan\beta} \label{equ:trig-10}
\end{gather}

\paragraph{倍角公式}
通过将和差角公式中的$\alpha\pm\beta$替换成$\alpha+\alpha=2\alpha$,可得到倍角公式
\begin{gather}
    \sin2\alpha=2\sin\alpha\cos\alpha \label{equ:trig-11} \\
    \cos2\alpha=\cos^2\alpha-\sin^2\alpha=2\cos^2\alpha-1=1-2\sin^2\alpha \label{equ:trig-12} \\
    \tan2\alpha=\frac{2\tan\alpha}{1-\tan^2\alpha} \label{equ:trig-13}
\end{gather}
公式\eqref{equ:trig-12}的三连等是通过公式\eqref{equ:trig-1}得来的。

\paragraph{降幂公式}
降幂公式通过倍角公式\eqref{equ:trig-11}和\eqref{equ:trig-12}得到
\begin{gather}
    \sin\alpha\cos\alpha=\frac{1}{2}\sin2\alpha \label{equ:trig-14} \\
    \sin^2\alpha=\frac{1-\cos2\alpha}{2} \label{equ:trig-15} \\
    \cos^2\alpha=\frac{1+\cos2\alpha}{2} \label{equ:trig-16}
\end{gather}

\paragraph{半角公式}
通过给降幂公式代入$\dfrac{\alpha}{2}$并开根,即可得到半角公式
\begin{gather}
    \sin\frac{\alpha}{2}=\pm\sqrt{\frac{1-\cos\alpha}{2}} \label{equ:trig-17} \\
    \cos\frac{\alpha}{2}=\pm\sqrt{\frac{1+\cos\alpha}{2}} \label{equ:trig-18} \\
    \tan\frac{\alpha}{2}=\pm\sqrt{\frac{1+\cos\alpha}{1-\cos\alpha}}=\frac{\sin\alpha}{1+\cos\alpha}=\frac{1-\cos\alpha}{\sin\alpha} \label{equ:trig-19}
\end{gather}
公式\eqref{equ:trig-19}可巧记为“$\sin$在一边,母加子减”。

\paragraph{万能公式}
也叫\textbf{正切半角公式}(tangent half-angle formula),它将函数名都统一为$\tan$且角都统一成$\dfrac{\alpha}{2}$
\begin{gather}
    \tan\frac{\alpha}{2}=\frac{2\tan\frac{\alpha}{2}}{1-\tan^2\frac{\alpha}{2}} \label{equ:trig-20} \\
    \sin\frac{\alpha}{2}=\frac{2\tan\frac{\alpha}{2}}{1+\tan^2\frac{\alpha}{2}} \label{equ:trig-21} \\
    \cos\frac{\alpha}{2}=\frac{1-\tan^2\frac{\alpha}{2}}{1+\tan^2\frac{\alpha}{2}} \label{equ:trig-22}
\end{gather}
“万能公式”当然很有用,只不过在高中时期还没有显现它的价值罢了。

% TODO: \subsection{诱导公式}

\subsection{辅助角公式}
这个公式也是非常重要的,对于正弦函数和余弦函数的线性组合,有\[a\sin x+b\cos x=\sqrt{a^2+b^2}\sin(x+\varphi)\quad(a>0)\]

其中\[\varphi=\arctan\frac{b}{a}\]

对于$a<0$的情况,$\varphi=\arctan\dfrac{b}{a}\pm\pi$(加还是减随意,不想看到$\pi$的话在根号前添负号也可以)。

% TODO: \section{化简}
