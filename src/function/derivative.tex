\chapter{导数}
在此首先恭喜您,当看到“导数”这个词时,阁下即将面对的是数学的一个全新的分支---\textbf{微积分学}(calculus)。简而言之,它将我们研究函数的视角从宏观拓展到了微观。

听起来似乎很难,不过这只是高中难度的知识,我们也只学习导数及其应用,也没这么难吧(应该)。

YouTube频道3Blue1Brown有10部讲述“微积分的本质”的视频,我会按学习进度推荐您看其中的6部。

\section{概念}
有一描述直线的函数$f(x)=2x$,当问及“$x=1$时函数的斜率是多少”,您一定会说是2。这很简单,因为$f(x)$的斜率处处相等。

那就再问一个进阶的问题:当$x=1$时,$f(x)=x^2$的斜率为多少呢?

作为一个不知道微积分为何物的高中生,您决定试一试,透过斜率的定义\footnote{简单来说,斜率$k=\dfrac{\Delta y}{\Delta x}$,即垂直方向的变化量与水平方向的变化量的比值。}来解决这个问题。

\begin{figure}[htb]
    \centering
    \begin{asy}
        import graph;
        write("asy: Generating 'fig:calc-slope'");

        size(4cm, 4cm);

        real f(real x){return x^2;}
        real fp(real x){return 2.5*x-1.5;}

        draw(graph(f, -2, 2), red);
        draw(graph(fp, 0, 2), blue);
        xaxis("$x$", EndArrow);
        yaxis("$y$", EndArrow);

        draw((1, 1) -- (1.5, 1), L=scale(0.5)*Label("$\Delta x$", position=MidPoint));
        draw((1.5, 1) -- (1.5, 1.5^2), L=scale(0.5)*Label("$\Delta y$", position=MidPoint));
        dot((1, 1), L=scale(0.5)*Label("$A$", align=NW));
        dot((1.5, 1.5^2), L=scale(0.5)*Label("$B$", align=W));
    \end{asy}
    \caption{计算二次函数的斜率}
    \label{fig:calc-slope}
\end{figure}

您或许会画出如图\ref{fig:calc-slope}所示的函数图像,那接下来我们就可以使用由两条直角边$\Delta x$和$\Delta y$组成的直角三角形来计算斜率。

二次函数上$A$点处的斜率是我们要求的,我们可以让抛物线上的$B$点慢慢靠近$A$点,这样直角三角形会越来越小,结果也会变得精确。我们就可以写出式子\eqref{equ:calc-slope}来表示斜率
\begin{gather}
    k=\frac{(1+\Delta x)^2-1^2}{\Delta x} \label{equ:calc-slope}
\end{gather}

给$\Delta x$多次赋值,我们不难发现最后它会收敛到2,由此可知$f(x)=x^2$在$x=1$处的斜率是2。
\begin{align*}
    \Delta x=0.1\quad & k=2.1 \\
    \Delta x=0.01\quad & k=2.01 \\
    \Delta x=0.001\quad & k=2.001 \\
\end{align*}

好,我们算完了,感觉很简单吧!\textbf{导数}(derivative)就是这样,研究函数在某一点附近的变化率。我们刚刚寻找函数某点导数的过程称为\textbf{求导}(differentiation)。

同时,为了避免每次求导都要考虑函数图像,求$f(x)$在$x=x_0$处的导数$f'(x_0)$有公式\[f'(x_0)=\lim_{h\to0}\frac{f(x_0+h)-f(x_0)}{h}\]这和我们刚刚得到的式子\eqref{equ:calc-slope}很像,只不过我们使用了极限的表示方法。

% TODO: \section{公式}
% TODO: \section{运用}
