\chapter{导数}
在此首先恭喜您,当看到“导数”这个词时,阁下即将面对的是数学的一个全新的分支---\textbf{微积分学}(calculus)。简而言之,它将我们研究函数的视角从宏观拓展到了微观。

听起来似乎很难,不过这只是高中难度的知识,我们也只学习导数及其应用,也没这么难吧(应该)。

YouTube频道3Blue1Brown有10部讲述“微积分的本质”的视频,我会按学习进度推荐您看其中的6部。

\section{概念}
有一描述直线的函数$f(x)=2x$,当问及“$x=1$时函数的斜率是多少”,您一定会说是2。这很简单,因为$f(x)$的斜率处处相等。

那就再问一个进阶的问题:当$x=1$时,$f(x)=x^2$的斜率为多少呢?

作为一个不知道求导为何物的高中生,您决定试一试,透过斜率的定义\footnote{简单来说,斜率$k=\dfrac{\Delta y}{\Delta x}$,即垂直方向的变化量与水平方向的变化量的比值。}来解决这个问题。

\begin{figure}[htb]
    \centering
    \begin{asy}
        import graph;
        write("asy: Generating 'fig:calc-slope'");

        size(4cm, 4cm);

        real f(real x){return x^2;}
        real fp(real x){return 2.5*x-1.5;}

        draw(graph(f, -2, 2), red);
        draw(graph(fp, 0, 2), blue);
        xaxis("$x$", EndArrow);
        yaxis("$y$", EndArrow);

        draw((1, 1) -- (1.5, 1), L=scale(0.5)*Label("$\Delta x$", position=MidPoint));
        draw((1.5, 1) -- (1.5, 1.5^2), L=scale(0.5)*Label("$\Delta y$", position=MidPoint));
        dot((1, 1), L=scale(0.5)*Label("$A$", align=NW));
        dot((1.5, 1.5^2), L=scale(0.5)*Label("$B$", align=W));
    \end{asy}
    \caption{计算二次函数的斜率}
    \label{fig:calc-slope}
\end{figure}

您或许会画出如图\ref{fig:calc-slope}所示的函数图像,那接下来我们就可以使用由两条直角边$\Delta x$和$\Delta y$组成的直角三角形来计算斜率。

二次函数上$A$点处的斜率是我们要求的,我们可以让抛物线上的$B$点慢慢靠近$A$点,这样直角三角形会越来越小,结果也会变得精确。我们就可以写出式子\eqref{equ:calc-slope}来表示斜率
\begin{gather}
    k=\frac{(1+\Delta x)^2-1^2}{\Delta x} \label{equ:calc-slope}
\end{gather}

给$\Delta x$多次赋值,我们不难发现最后它会收敛到2,不难得出$f(x)=x^2$在$x=1$处的斜率是2。
\begin{gather*}
    \Delta x=0.1 \quad k=2.1 \\
    \Delta x=0.01 \quad k=2.01 \\
    \Delta x=0.001 \quad k=2.001
\end{gather*}

好,我们算完了,感觉很简单吧!\textbf{导数}(derivative)就是这样,研究函数在某一点附近的变化率。我们刚刚寻找函数某点导数的过程称为\textbf{求导}(differentiation)。

同时,为了避免每次求导都要考虑函数图像,求$f(x)$在$x=x_0$处的导数$f'(x_0)$有公式
\begin{gather}
    f'(x_0)=\lim_{h\to0}\frac{f(x_0+h)-f(x_0)}{h} \label{equ:fpx0}
\end{gather}
这和我们刚刚得到的式子\eqref{equ:calc-slope}很像,只不过我们使用了极限的表示方法。

我们采用在函数的右上角加上一短撇作为导数的记号,这是18世纪时拉格朗日发明的,比较方便。不过,莱布尼茨的记法($\dfrac{\mathrm{d}y}{\mathrm{d}x}$及其变种)也同样直观,它便于记忆导数计算的法则,我们将在之后介绍。

通常情况下,函数在定义域中任意一处都存在导数,导数值随着自变量的变化可看作是函数。把公式\eqref{equ:fpx0}稍加改变,则$f(x)$的导函数$f'(x)$可如下计算
\begin{gather}
    f'(x)=\lim_{h\to0}\frac{f(x+h)-f(x)}{h} \label{equ:fpx}
\end{gather}

\begin{example}
    求$f(x)=x^3$的导函数。
\end{example}

\begin{proof}[解]
    我们直接代入公式\eqref{equ:fpx}即可
    \begin{align*}
        f'(x)&=\lim_{h\to0}\frac{(x+h)^3-x^3}{h} \\
             &=\lim_{h\to0}\frac{3hx^2+3h^2x+h^3}{h} \\
             &=\lim_{h\to0}3x^2+3hx+h^2 \\
             &=3x^2
    \end{align*}

    如果对以上求极限的操作感到疑惑的话,我们要求出当$h$趋近于0时一个式子的值,不过这会出现“除以0”的问题,所以进行了这样的操作(不管听没听懂,往下看就是了)。

    以上,我们得出$(x^3)'=3x^2$\qedhere
\end{proof}

最后,我想讲一下“求导”和“微分”的区别。虽然在一些情况下它俩的含义相同,不过“微分”是一种思想,例如把函数图像看成是无数小线段的集合;“求导”则是一种运算,对函数的增减性进行定量描述。

\section{公式及运算}
在之前,我们了解了导数是如何计算的,还求了$f(x)=x^3$的导函数。如果不怕麻烦,我们也可以求出另外一些函数的导函数,不过还有一些基础函数(如$\sin x$和$\log_ax$)的导函数以目前的水平还无能为力,不过它们的导数往往也是很重要的。

以下,是基础函数的导数(好心提示:这些公式要背的)
\begin{align}
    (C)'&=0 & (x^a)'&=ax^{a-1} \label{equ:df-1} \\
    (a^x)'&=a^x\ln a & (\log_ax)'&=\frac{1}{x\ln a} \label{equ:df-2} \\
    (\sin x)'&=\cos x & (\cos x)'&=-\sin x \label{equ:df-3}
\end{align}

对于公式\eqref{equ:df-2}来说,当$a=e$时可将其作为特殊情况记忆
\begin{align*}
    (e^x)'&=e^x & (\ln x)'=\frac{1}{x}
\end{align*}

导数的四则运算与代数相比差异是巨大的
\begin{gather}
    \big(af(x)+bg(x)\big)'=af'(x)+bg(x)' \label{equ:drule-1} \\
    \big(f(x)g(x)\big)'=f'(x)g(x)+f(x)g'(x) \label{equ:drule-2} \\
	\left(\frac{f(x)}{g(x)}\right)'=\frac{f'(x)g(x)-f(x)g'(x)}{g^2(x)} \label{equ:drule-3}
\end{gather}

% TODO: \section{运用}
