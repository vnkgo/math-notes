\chapter{计数原理}
在“计数原理”这一章节中,我们主要学习数数。不过之前我们使用的是穷举法,这实在是太麻烦了。所以我们引入了乘法原理和加法原理两种基本的计数原理,并且使用排列组合来选和/或排数量众多的元素,最后总结了常见的题型。

\section[乘法原理]{乘法原理(分步)}
\textbf{乘法原理}(rule of product或multiplication principle)简而言之是“完成一件事要依次完成$n$个步骤,第一步有$a_1$种方法,第二步有$a_2$种方法,\ldots\ldots,第$n$步有$a_n$种方法,则完成这件事共有$a_1\times a_2 \times\cdots\times a_n$种方法”。

\section[加法原理]{加法原理(分类)}
与乘法原理不同,\textbf{加法原理}(rule of sum或addition principle)是“完成一件事有$n$类不同的方法,第一类有$a_1$种方法,第二类有$a_2$种方法,\ldots\ldots,第$n$类有$a_n$种方法,则完成这件事共有$a_1+a_2+\cdots+a_n$种方法”。

虽然乘法原理和加法原理的叙述很像,但它们是本质不同的两种方法:乘法原理的每一步(一个乘数)是一个步骤,整件事并没有完成;加法原理的每一类(一个加数)是完成这件事的方法数,整件事已经完成。

所以加法原理可以看作是我们使用乘法原理计算发现前步的选择对后步的可能性种数产生影响时,对各种情况进行分类讨论的一种方法

\subsection{容斥原理}
\textbf{容斥原理}(inclusion–exclusion principle)则可以看作是加法原理的拓展

\section{排列数和组合数}
在计算可能性种数的时候,与$10\times 9\times 8$类似的运算重复出现。如果相乘的数字增加,式子会变得越来越长,我们需要一种方法来简化表示方法。

因此,我们引入\textbf{排列数}(permutation),表示从$m$个元素中有顺序地选出$n$个元素的可能性种数,记作$P_m^n$,计算它的公式为\[P_m^n=\frac{m!}{(m-n)!}\]其中$n!$表示$n$的\textbf{阶乘}(factorial),表示所有小于等于该数的正整数的积,如$5!=5\times 4\times 3\times 2\times 1=120$,同时规定$0!=1$。

为了简化表示,以下的写法是等价的,这被称为“$n$的全排”。\[P_n^n=P_n=n!\]

从$m$个元素中取出(并不排序)$n$个元素称为\textbf{组合数}(combination),记作$C_m^n$,具体的公式为\[C_m^n=\frac{m!}{n!(m-n)!}\]

根据排列数的概念,$P_m^n$可写成$C_m^nP_n$,而$C_m^n$是$\dfrac{P_m^n}{n!}$

同时,对于排列数来说还有几个有用的公式
\begin{gather}
	C_n^r=C_n^{n-r} \label{equ:comb-1} \\
	C_n^0+C_n^1+\cdots+C_n^n=2^n \label{equ:comb-2} \\
	C_m^n+C_m^{n+1}=C_{m+1}^{n+1} \label{equ:comb-3}
\end{gather}
公式\eqref{equ:comb-1}可以理解为“从10个元素中取4个元素”与“从10个元素中留下6个元素”的可能性种数是相等的;而公式\eqref{equ:comb-3}可简记为“下加一,上取大”。

\begin{example}
	$C_n^3+C_n^4=C_{n+1}^6$,且$n\geq5$,求$n$。
\end{example}

\begin{proof}[解]
	可先用公式\eqref{equ:comb-3}变形为\[C_{n+1}^4=C_{n+1}^6\]

	可以套用公式\eqref{equ:comb-1}得方程\[n+1=4+6\]

	解得\[n=9\qedhere\]
\end{proof}

\section{常见题型}
最基本的计数原理也就是乘法原理和加法原理两个,灵活运用它们即可解决所有问题.不过对于某些有着典型特征的题目,这里也总结了做法。

\subsection{打包法}
我们时常会看到类似于“\ldots\ldots 要在一起”这样的要求,这个时候我们要将可以在一起的元素看成一整个元素进行排列。

\begin{example}
	$3$名男生和$4$名女生排队,女生要排在一起,问有多少种排法?
\end{example}

\subsection{插入法}
有时候也会出现“\ldots\ldots 不能在一起”的奇葩限制,我们也可以使用打包法将无限制元素打包,再将不能在一起的元素插入包中元素。

\begin{example}
	$3$名女生和$4$名男生排队,女生不能排在一起,问有多少种排法?
\end{example}

\begin{proof}[解]
	如题,“女生不能排在一起”的最简单的方法就是先排男生,然后将女生插入男生之间的空位中。

	\begin{itemize}
		\item 先找插入位,有$C_5^3$种可能
		\item 再排序要插入的元素(女生),有$P_3$种可能
		\item 最后排无限制的元素(男生),有$P_4$种可能
	\end{itemize}

	故一共有$C_5^3P_3P_4=1440$种排法。
\end{proof}

更有甚者,会有“甲、乙、丙三人中的两人可以排在一起但是三人不能同时排在一起”的类似“修罗场”的关系,也可以使用插入法解决。

\begin{example}
	$6$人排队,甲、乙、丙三人中的两人可以排在一起但是三人不能同时排在一起,问有多少种排法?
\end{example}

\subsection{插板法}
在另一些情况下我们被要求“将一些\uline{相同的}物件分给几个人,一个人\uline{至少}分到一个”,可使用\textbf{插板法}(stars and bars)来解决。

有如下5个相同元素,要将其分成3组,可看成在4个空位选2个插入两块板子得到三组,即$C_4^2$种分法。

\begin{center}
	\verb*|o o o o o|
\end{center}

根据此种方法,将$m$个相同的元素分成$n$组,且必须分完、不出现空组的分法有\[C_{m-1}^{n-1}\]

如果允许出现空组的话,我们可以加上组数个元素再插板,被分到一个实际上是什么也没有. 这样共有$C_{m+n-1}^{n-1}$种分法。

\subsection{至多至少问题}

\subsection{混合分组}

\section[概率]{计数原理在古典概率中的应用}
