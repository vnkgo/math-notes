\section*{前言}
对数学笔记的数字化工作很久之前就在盘算了,苦于种种原因没有实现。从高一的寒假开始计划,直到高一快结束了才开始提笔。

\begin{quote}
	非常感谢2022年3月上海市2019冠状病毒病聚集性疫情,在家里的几个月中净想着摸鱼了,结果整出了一堆奇奇怪怪的东西。
\end{quote}

或许我并不适合总结这一堆笔记,我的数学只是勉强算还可以吧。但我不整理又有谁整理呢?大家还不是高考后把书啊、笔记啊都丢掉或卖掉赚钱,完全没有想过给自己(或自己的后代)留下一点东西。

这本笔记的诞生首先要感谢我的数学老师,他使用了非常独特的教学方法:使用\textbf{思维导图}来授课,这能很清晰的反应了各个知识点之间的逻辑关系。不过使用\LaTeX{}实现起来有困难,就算了吧。这同时也是数字化的一大难点:思维导图只保留了精华,而放弃了一些有助于理解的说明。这就是这本笔记要补充的内容。

如此这般这本笔记就完成了,不过还存在着一个大问题。你翻到目录,发现数理分析章节在代数基础后面。按照常理,你会先翻阅代数基础,然后发现有些地方不能理解。这个很好解释:每一章的内容只有相关性,不一定有连续性。这是有意为之的,否则就和教科书没有区别了。所以说请按需索取。

这本笔记是开源的,您可以在{\color{blue}\href{https://github.com/jason-bowen-zheng/math-notes}{这里}}获取全部的源代码。

\begin{flushright}
	\date{2022年6月}于上海
\end{flushright}
