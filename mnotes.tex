\documentclass[a4paper, openany, UTF8]{ctexbook}

\usepackage{asymptote}
\usepackage{amsmath}
\usepackage{amssymb}
\usepackage{amsthm}
\usepackage{float}
\usepackage{geometry}
\usepackage{graphicx}
\usepackage[colorlinks=true]{hyperref}
\usepackage[toc]{multitoc}
\usepackage{paralist}
\usepackage{subcaption}
\usepackage{ulem}
\usepackage{wrapfig2}
\usepackage{xcolor}

\setlength{\parskip}{1ex plus 0.5ex minus 0.2ex}
\geometry{inner=1in,outer=1.25in}

\newtheorem{axiom}{公理}[chapter]
\newtheorem{corollary}{推论}[chapter]
\newtheorem{example}{例}[section]
\newtheorem{lexample}{引例}[chapter]
\newtheorem{theorem}{定理}[chapter]

\let\enumerate\compactenum
\let\endenumerate\endcompactenum
\let\itemize\compactitem
\let\enditemize\endcompactitem

\title{{\Huge 高中数学笔记}}
\author{郑}

\begin{document}
	\begin{asydef}
		import settings;
		batchMask = false;
		interactiveMask = true;
		batchView = false;
		interactiveView = true;
		prc = false;
	\end{asydef}

	\frontmatter
	\setcounter{tocdepth}{1}
	\maketitle

	\section*{前言}
简而言之,该数学笔记整理了高中数学的所有知识点,供各位同学参考。它有点像教科书,但又不是教科书,这一点会在正文中慢慢体现出来。

这本笔记分成了如下部分:

\begin{description}
	\item[代数] 包括了高中代数这一纷繁杂乱的东西
	\item[函数] 涵盖了函数图像、单调性、对称性、周期性以及导数
	\item[三角学] 和三角比、三角函数以及三角形有关的东西
	\item[向量和复数] 向量和复数的相关知识
	\item[立体几何] 顾名思义,是三维的几何
	\item[计数原理] 排列组合和二项式的知识
	\item[概率和统计] 赌徒专用(雾)
\end{description}

该数学笔记亦提供了计算器的使用说明,适用于卡西欧\verb|fx-991CN X|。

同时,数学的学习不能局限于单一来源,我也会时不时附上一些清晰易懂的文章或视频的链接。

这本笔记是开源的,并且采用\href{https://creativecommons.org/licenses/by/4.0}{知识共享署名 4.0 国际许可协议}进行许可。您可以在\href{https://github.com/jason-bowen-zheng/math-notes}{这里}获取全部的源代码,亦可以提出改进意见。
\hypersetup{hidelinks}

\begin{flushright}
	\date{2022年11月}于上海
\end{flushright}


	\tableofcontents
	\clearpage
	\mainmatter
	\raggedbottom

	\part{函数}
	\chapter{导数}
在此首先恭喜您,当看到“导数”这个词时,阁下即将面对的是数学的一个全新的分支---\textbf{微积分学}(calculus)。简而言之,它将我们研究函数的视角从宏观拓展到了微观。

听起来似乎很难,不过这只是高中难度的知识,我们也只学习导数及其应用,也没这么难吧(应该)。

YouTube频道3Blue1Brown有10部讲述“微积分的本质”的视频,我会按学习进度推荐您看其中的6部。

\section{概念}
有一描述直线的函数$f(x)=2x$,当问及“$x=1$时函数的斜率是多少”,您一定会说是2。这很简单,因为$f(x)$的斜率处处相等。

那请问当$x=1$时,$f(x)=x^2$的斜率为多少呢?

作为一个不知道微积分为何物的高中生,您决定试一试,透过斜率的定义\footnote{简单来说,斜率$k=\dfrac{\Delta y}{\Delta x}$,即垂直方向的变化量与水平方向的变化量的比值。}来解决这个问题。

\begin{figure}[htb]
    \centering
    \begin{asy}
        import graph;
        write("asy: Generating 'fig:calc-slope'");

        size(4cm, 4cm);

        real f(real x){return x^2;}
        real fp(real x){return 2.5*x-1.5;}

        draw(graph(f, -2, 2), red);
        draw(graph(fp, 0, 2), blue);
        xaxis("$x$", EndArrow);
        yaxis("$y$", EndArrow);

        draw((1, 1) -- (1.5, 1), L=scale(0.5)*Label("$\Delta x$", position=MidPoint));
        draw((1.5, 1) -- (1.5, 1.5^2), L=scale(0.5)*Label("$\Delta y$", position=MidPoint));
        dot((1, 1), L=scale(0.5)*Label("$A$", align=NW));
        dot((1.5, 1.5^2), L=scale(0.5)*Label("$B$", align=W));
    \end{asy}
    \caption{计算二次函数的斜率}
    \label{fig:calc-slope}
\end{figure}

您或许会画出如图\ref{fig:calc-slope}所示的函数图像,那接下来我们就可以使用由两条直角边$\Delta x$和$\Delta y$组成的直角三角形来计算斜率。

二次函数上$A$点处的斜率是我们要求的,我们可以让抛物线上的$B$点慢慢靠近$A$点,这样直角三角形会越来越小,结果也会变得精确。我们就可以写出式子\eqref{equ:calc-slope}来表示斜率
\begin{gather}
    k=\frac{(1+\Delta x)^2-1^2}{\Delta x} \label{equ:calc-slope}
\end{gather}

给$\Delta x$多次赋值,我们不难发现最后它会收敛到2,由此可知$f(x)=x^2$在$x=1$处的斜率是2。
\begin{align*}
    \Delta x=0.1\quad & k=2.1 \\
    \Delta x=0.01\quad & k=2.01 \\
    \Delta x=0.001\quad & k=2.001 \\
\end{align*}

好,我们算完了,感觉很简单吧!\textbf{导数}(derivative)就是这样,研究函数在某一点附近的变化率。我们刚刚寻找函数某点导数的过程称为\textbf{求导}(differentiation)。

同时,为了避免每次求导都要考虑函数图像,求$f(x)$在$x=x_0$处的导数$f'(x_0)$有公式\[f'(x_0)=\lim_{h\to0}\frac{f(x_0+h)-f(x_0)}{h}\]这和我们刚刚得到的式子\eqref{equ:calc-slope}很像,只不过我们使用了极限的表示方法。

% TODO: \section{公式}
% TODO: \section{运用}


	\part{三角学}
	\chapter{三角比}
在初中时期,我们通过直角三角形定义并明白了锐角三角比。现在,我们将要学习任意角的三角比。

\section{概念}

\subsection{弧度制}
在之前很长一段时间,我们使用的都是角度制,它将一个周角分为了360等分,不过继续运用角度制会显得不够简洁。所以我们引入了\textbf{弧度}(radian),它认为一个周角是$2\pi$弧度。

从角度制转换为弧度制很简单,乘以$\dfrac{\pi}{180}$即可;以弧度表示角度,则乘以$\dfrac{180}{\pi}$。所以$1\mathrm{rad}=57.296^\circ$。

弧度的单位是$\mathrm{rad}$,不过通常不写单位。

\subsection{任意角的定义}

\subsection{三角比的定义}

\section{公式}

\section{化简}

	\chapter{三角函数}

	\part{计数原理}
	\chapter{计数原理}
在“计数原理”这一章节中,我们主要学习数数。不过之前我们使用的是穷举法,这实在是太麻烦了。所以我们引入了乘法原理和加法原理两种基本的计数原理,并且使用排列组合来选和/或排数量众多的元素,最后总结了常见的题型。

\section[乘法原理]{乘法原理(分步)}
\textbf{乘法原理}(rule of product或multiplication principle)简而言之是“完成一件事要依次完成$n$个步骤,第一步有$a_1$种方法,第二步有$a_2$种方法,\ldots\ldots,第$n$步有$a_n$种方法,则完成这件事共有$a_1\times a_2 \times\cdots\times a_n$种方法”。

\section[加法原理]{加法原理(分类)}
与乘法原理不同,\textbf{加法原理}(rule of sum或addition principle)是“完成一件事有$n$类不同的方法,第一类有$a_1$种方法,第二类有$a_2$种方法,\ldots\ldots,第$n$类有$a_n$种方法,则完成这件事共有$a_1+a_2+\cdots+a_n$种方法”。

虽然乘法原理和加法原理的叙述很像,但它们是本质不同的两种方法:乘法原理的每一步(一个乘数)是一个步骤,整件事并没有完成;加法原理的每一类(一个加数)是完成这件事的方法数,整件事已经完成。

所以加法原理可以看作是我们使用乘法原理计算发现前步的选择对后步的可能性种数产生影响时,对各种情况进行分类讨论的一种方法

\subsection{容斥原理}
\textbf{容斥原理}(inclusion–exclusion principle)则可以看作是加法原理的拓展

\section{排列数和组合数}
在计算可能性种数的时候,与$10\times 9\times 8$类似的运算重复出现。如果相乘的数字增加,式子会变得越来越长,我们需要一种方法来简化表示方法。

因此,我们引入\textbf{排列数}(permutation),表示从$m$个元素中有顺序地选出$n$个元素的可能性种数,记作$P_m^n$,计算它的公式为\[P_m^n=\frac{m!}{(m-n)!}\]其中$n!$表示$n$的\textbf{阶乘}(factorial),表示所有小于等于该数的正整数的积,如$5!=5\times 4\times 3\times 2\times 1=120$,同时规定$0!=1$。

为了简化表示,以下的写法是等价的,这被称为“$n$的全排”。\[P_n^n=P_n=n!\]

从$m$个元素中取出(并不排序)$n$个元素称为\textbf{组合数}(combination),记作$C_m^n$,具体的公式为\[C_m^n=\frac{m!}{n!(m-n)!}\]

根据排列数的概念,$P_m^n$可写成$C_m^nP_n$,而$C_m^n$是$\dfrac{P_m^n}{n!}$

同时,对于排列数来说还有几个有用的公式
\begin{gather}
	C_n^r=C_n^{n-r} \label{equ:comb-1} \\
	C_n^0+C_n^1+\cdots+C_n^n=2^n \label{equ:comb-2} \\
	C_m^n+C_m^{n+1}=C_{m+1}^{n+1} \label{equ:comb-3}
\end{gather}
公式\eqref{equ:comb-1}可以理解为“从10个元素中取4个元素”与“从10个元素中留下6个元素”的可能性种数是相等的;而公式\eqref{equ:comb-3}可简记为“下加一,上取大”。

\begin{example}
	$C_n^3+C_n^4=C_{n+1}^6$,且$n\geq5$,求$n$。
\end{example}

\begin{proof}[解]
	可先用公式\eqref{equ:comb-3}变形为\[C_{n+1}^4=C_{n+1}^6\]

	可以套用公式\eqref{equ:comb-1}得方程\[n+1=4+6\]

	解得\[n=9\qedhere\]
\end{proof}

\section{常见题型}
最基本的计数原理也就是乘法原理和加法原理两个,灵活运用它们即可解决所有问题.不过对于某些有着典型特征的题目,这里也总结了做法。

\subsection{打包法}
我们时常会看到类似于“\ldots\ldots 要在一起”这样的要求,这个时候我们要将可以在一起的元素看成一整个元素进行排列。

\begin{example}
	$3$名男生和$4$名女生排队,女生要排在一起,问有多少种排法?
\end{example}

\subsection{插入法}
有时候也会出现“\ldots\ldots 不能在一起”的奇葩限制,我们也可以使用打包法将无限制元素打包,再将不能在一起的元素插入包中元素。

\begin{example}
	$3$名女生和$4$名男生排队,女生不能排在一起,问有多少种排法?
\end{example}

\begin{proof}[解]
	如题,“女生不能排在一起”的最简单的方法就是先排男生,然后将女生插入男生之间的空位中。

	\begin{itemize}
		\item 先找插入位,有$C_5^3$种可能
		\item 再排序要插入的元素(女生),有$P_3$种可能
		\item 最后排无限制的元素(男生),有$P_4$种可能
	\end{itemize}

	故一共有$C_5^3P_3P_4=1440$种排法。
\end{proof}

更有甚者,会有“甲、乙、丙三人中的两人可以排在一起但是三人不能同时排在一起”的类似“修罗场”的关系,也可以使用插入法解决。

\begin{example}
	$6$人排队,甲、乙、丙三人中的两人可以排在一起但是三人不能同时排在一起,问有多少种排法?
\end{example}

\subsection{插板法}
在另一些情况下我们被要求“将一些\uline{相同的}物件分给几个人,一个人\uline{至少}分到一个”,可使用\textbf{插板法}(stars and bars)来解决。

有如下5个相同元素,要将其分成3组,可看成在4个空位选2个插入两块板子得到三组,即$C_4^2$种分法。

\begin{center}
	\verb*|o o o o o|
\end{center}

根据此种方法,将$m$个相同的元素分成$n$组,且必须分完、不出现空组的分法有\[C_{m-1}^{n-1}\]

如果允许出现空组的话,我们可以加上组数个元素再插板,被分到一个实际上是什么也没有. 这样共有$C_{m+n-1}^{n-1}$种分法。

\subsection{至多至少问题}

\subsection{混合分组}

\section[概率]{计数原理在古典概率中的应用}

	\chapter{二项式定理}
相比于之前的乘法原理和加法原理,二项式定理比较简单,它主要研究形如$(a+b)^n$的展开问题。

\section{公式}
我们可以先从简单的问题开始思考起来。

\begin{lexample}
	$(a+b)^3$的展开式是什么?
\end{lexample}

利用初中的知识,将上式变成$(a+b)(a+b)(a+b)$并展开可得\[(a+b)^3=a^3+3a^2b+3ab^2+b^3\]

\end{document}
